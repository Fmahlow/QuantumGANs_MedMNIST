%Version 3.1 December 2024
% See section 11 of the User Manual for version history
%
%%%%%%%%%%%%%%%%%%%%%%%%%%%%%%%%%%%%%%%%%%%%%%%%%%%%%%%%%%%%%%%%%%%%%%
%%                                                                 %%
%% Please do not use \input{...} to include other tex files.       %%
%% Submit your LaTeX manuscript as one .tex document.              %%
%%                                                                 %%
%% All additional figures and files should be attached             %%
%% separately and not embedded in the \TeX\ document itself.       %%
%%                                                                 %%
%%%%%%%%%%%%%%%%%%%%%%%%%%%%%%%%%%%%%%%%%%%%%%%%%%%%%%%%%%%%%%%%%%%%%

\documentclass[pdflatex,sn-mathphys-num]{sn-jnl}

% Standard packages
\usepackage{graphicx}
\usepackage{multirow}
\usepackage{amsmath,amssymb,amsfonts}
\usepackage{amsthm}
\usepackage{mathrsfs}
\usepackage[title]{appendix}
\usepackage{xcolor}
\usepackage{textcomp}
\usepackage{manyfoot}
\usepackage{booktabs}
\usepackage{algorithm}
\usepackage{algorithmicx}
\usepackage{algpseudocode}
\usepackage{listings}

\theoremstyle{thmstyleone}
\newtheorem{theorem}{Theorem}
\newtheorem{proposition}[theorem]{Proposition}

\theoremstyle{thmstyletwo}
\newtheorem{example}{Example}
\newtheorem{remark}{Remark}

\theoremstyle{thmstylethree}
\newtheorem{definition}{Definition}

\raggedbottom

\begin{document}

\title[Classical vs Quantum GANs for BreastMNIST]{Benchmarking Classical and Quantum Generative Models for Breast Ultrasound Synthesis}

\author*[1]{\fnm{First} \sur{Author}}\email{author1@example.com}
\author[1]{\fnm{Second} \sur{Author}}\email{author2@example.com}
\author[1]{\fnm{Third} \sur{Author}}\email{author3@example.com}
\affil*[1]{\orgdiv{Department}, \orgname{Institution}, \orgaddress{\city{City}, \country{Country}}}

\abstract{Limited availability of labeled medical images motivates the use of synthetic data to improve supervised learning and preserve privacy. We benchmark classical and quantum generative adversarial networks (GANs) on the BreastMNIST dataset, comparing Conditional GAN (CGAN), Deep Convolutional GAN (DCGAN), and Wasserstein GAN with gradient penalty (WGAN-GP) against quantum counterparts: QPatchGAN, MOSAIQ, and Last-QGAN. Models are evaluated using Fr\'echet Inception Distance (FID) and Inception Score (IS) for image fidelity and by measuring the impact of synthetic data on a ResNet-18 classifier. The study provides the first systematic comparison between classical and quantum GANs for breast ultrasound image synthesis in noisy intermediate-scale quantum settings.}

\keywords{Generative adversarial networks, quantum machine learning, synthetic medical data, BreastMNIST}

\maketitle

\section{Introduction}\label{sec:introduction}
Access to large, balanced, and well-annotated datasets remains a major challenge in medical imaging. Privacy regulations, acquisition costs, and class imbalance hinder the use of supervised learning, especially for rare conditions. Generative adversarial networks (GANs) have emerged as a solution for synthesizing realistic medical images, enabling data augmentation while preserving patient confidentiality. Recent advances in quantum computing introduce quantum GANs (QGANs), which leverage parametrized quantum circuits and are potentially suited for the limited data regime of noisy intermediate-scale quantum (NISQ) devices.

This work evaluates whether QGANs can rival established classical architectures when generating breast ultrasound images. Using the BreastMNIST dataset, we compare CGAN, DCGAN, and WGAN-GP with three quantum models: QPatchGAN, MOSAIQ, and Last-QGAN. Our contributions are:
\begin{itemize}
    \item A benchmark of classical and quantum models using FID and IS.
    \item An assessment of how synthetic data affects the training of a ResNet-18 classifier.
    \item A discussion of the practicality of QGANs for healthcare applications.
\end{itemize}

\section{Related Work}\label{sec:related}
GANs are widely used to generate medical images such as radiographs and histopathology slides, addressing data scarcity and class imbalance. Quantum generative models have recently been proposed, exploiting quantum parallelism to represent complex distributions with relatively few parameters. Hybrid approaches comparing classical and quantum pipelines remain rare, motivating a rigorous study that spans both paradigms.

\section{Methodology}\label{sec:method}
\subsection{Dataset}
BreastMNIST contains 780 ultrasound images labeled as normal, benign, or malignant. Following prior work, we merge normal and benign into a single positive class, forming a binary classification task. The images are converted to grayscale \(28\times28\) tensors and split into training, validation, and test sets with a 7:1:2 ratio. Standard normalization scales pixels to \([-1,1]\).

\subsection{Classical Models}
We implement three established GANs. DCGAN employs convolutional layers with batch normalization and ReLU activations. CGAN conditions the generator and discriminator on class labels through embedding layers. WGAN-GP minimizes the Wasserstein distance with gradient penalty to improve training stability.

\subsection{Quantum Models}
QPatchGAN extends PatchGAN with variational quantum circuits applied to image patches. MOSAIQ introduces mid-circuit measurements to recycle qubits, reducing hardware requirements. Last-QGAN uses hardware-efficient ans\"atze optimized with parameter shift gradients. All quantum models are trained in simulation with noise parameters reflecting NISQ devices.

\subsection{Evaluation Pipeline}
Generation quality is measured with FID and IS computed on 64-dimensional features from a pre-trained network. Impact on classification is examined by training a ResNet-18 with varying proportions \(r\) of synthetic data (\(r\in\{0,0.25,0.5,0.75,1.0\}\)) mixed with real samples. We report accuracy, precision, recall, F1-score, and area under the ROC curve (AUC). Finally, we inspect randomly generated images and two-dimensional embeddings for qualitative assessment.

\begin{table}[ht]
\caption{Summary of evaluated generative models.}
\centering
\begin{tabular}{lll}
\toprule
Model & Type & Key characteristics \\
\midrule
DCGAN & Classical & Convolutional generator and discriminator \\
CGAN & Classical & Label conditioning in both networks \\
WGAN-GP & Classical & Wasserstein loss with gradient penalty \\
QPatchGAN & Quantum & Patch-wise variational circuits \\
MOSAIQ & Quantum & Mid-circuit measurements for qubit reuse \\
Last-QGAN & Quantum & Hardware-efficient ansatz with parameter shift \\
\bottomrule
\end{tabular}
\label{tab:models}
\end{table}

\section{Results}\label{sec:results}
\subsection{Generation Quality}
Table~\ref{tab:fid} reports mean FID and IS scores. Classical models generally produce higher quality images, with CGAN achieving the best FID of 0.20. Quantum models exhibit larger FID values, reflecting current hardware and optimization limitations.

\begin{table}[ht]
\caption{Mean FID (lower is better) and IS (higher is better) for each generator.}
\centering
\begin{tabular}{lcc}
\toprule
Model & FID & IS \\
\midrule
DCGAN & 1.07 & 1.44 \\
CGAN & 0.20 & 1.63 \\
WGAN-GP & 2.79 & 1.59 \\
QPatchGAN & 6.63 & 1.32 \\
MOSAIQ & 6.63 & 1.32 \\
Last-QGAN & 6.63 & 1.32 \\
\bottomrule
\end{tabular}
\label{tab:fid}
\end{table}

\subsection{Impact on Classification}
Table~\ref{tab:classification} summarizes ResNet-18 performance when augmented with synthetic data. For DCGAN, moderate augmentation (\(r=0.5\)) yields the highest accuracy (0.87) and F1-score (0.91). Quantum-generated data provides comparable but slightly lower performance, with the best F1-score (0.90) at \(r=0.75\).

\begin{table}[ht]
\caption{Classification metrics for varying ratios \(r\) of synthetic to real data.}
\centering
\begin{tabular}{lcccccc}
\toprule
Model & \(r\) & Accuracy & Precision & Recall & F1 & AUC \\
\midrule
DCGAN & 0.0 & 0.81 & 0.90 & 0.83 & 0.86 & 0.79 \\
DCGAN & 0.25 & 0.72 & 0.91 & 0.68 & 0.78 & 0.75 \\
DCGAN & 0.50 & 0.87 & 0.87 & 0.96 & 0.91 & 0.78 \\
DCGAN & 0.75 & 0.85 & 0.88 & 0.93 & 0.90 & 0.79 \\
DCGAN & 1.00 & 0.83 & 0.83 & 0.96 & 0.89 & 0.72 \\
QPatchGAN & 0.0 & 0.83 & 0.89 & 0.87 & 0.88 & 0.79 \\
QPatchGAN & 0.25 & 0.82 & 0.86 & 0.89 & 0.88 & 0.76 \\
QPatchGAN & 0.50 & 0.81 & 0.82 & 0.94 & 0.88 & 0.70 \\
QPatchGAN & 0.75 & 0.84 & 0.85 & 0.95 & 0.90 & 0.75 \\
QPatchGAN & 1.00 & 0.80 & 0.87 & 0.86 & 0.86 & 0.75 \\
\bottomrule
\end{tabular}
\label{tab:classification}
\end{table}

\subsection{Visual Analysis}
Figure~\ref{fig:grid} shows randomly generated samples for each model. Classical generators produce sharper tumor boundaries, while quantum models yield more varied textures. Two-dimensional embeddings (Figure~\ref{fig:tsne}) indicate that quantum models cover a broader region of the feature space but deviate further from real data clusters.

\begin{figure}[ht]
\centering
\includegraphics[width=0.48\textwidth]{fig.eps}
\caption{Example images generated by classical (top) and quantum (bottom) models.}
\label{fig:grid}
\end{figure}

\begin{figure}[ht]
\centering
\includegraphics[width=0.48\textwidth]{fig.eps}
\caption{t-SNE projection of real and synthetic samples.}
\label{fig:tsne}
\end{figure}

\section{Discussion}\label{sec:discussion}
Classical GANs remain superior in FID and IS, benefiting from mature training techniques and larger network capacity. Quantum models demonstrate competitive classification impact despite poorer generation metrics, suggesting that diversity, rather than fidelity, drives augmentation gains. However, the high computational cost and noise sensitivity of current quantum hardware present practical barriers.

\section{Conclusion}\label{sec:conclusion}
We presented a comprehensive benchmark of classical and quantum GANs for breast ultrasound synthesis. CGAN achieved the best image quality, while quantum models provided modest improvements when augmenting classifier training. Future work includes hybrid quantum-classical architectures and evaluation on larger medical datasets.

\bibliography{sn-bibliography}

\end{document}
